\documentclass{article}
\usepackage[english]{babel}
\usepackage[utf8]{inputenc}
\usepackage[T1]{fontenc}
\usepackage{fancyhdr}%hlavicka
\usepackage{multirow,multicol}%zjednotene riadky,stlpce v tabulke
\usepackage{amsmath}%matematicke vzorce
\usepackage{longtable}%tabulka na viac stran

\input{karnaugh}%karnaughova mapa, obsahuje tikz library
\usetikzlibrary{automata,positioning,arrows}%prechodovy graf

%custom title
\usepackage{titling}
    \pretitle{\begin{center}\bfseries\fontsize{18bp}{18bp}\selectfont}
    \posttitle{\par\end{center}}
    \preauthor{}%\begin{center}\fontsize{14bp}{14bp}\selectfont}
    \author{}
    \postauthor{}%\par\end{center}\vspace{24bp}}
    \predate{}
    \date{}
    \postdate{}
    \setlength{\droptitle}{-50pt}

\title{Riešenie 3. zadania\\Syntéza sekvenčných logických obvodov}

%header definition
\pagestyle{fancy}
\fancyhf{}
\lhead{Martin Bodický}
\rhead{\today}
\cfoot{\thepage}
\renewcommand{\headrulewidth}{0.4pt}
\renewcommand{\footrulewidth}{0.4pt}

\begin{document}

%title
\maketitle
\thispagestyle{fancy}
%obsah
\section{Zadanie}
Navrhnite synchrónny sekvenčný obvod so vstupom x a výstupom y s nasledujúcim 
správaním: na výstupe y bude 1 vždy vtedy, ak sa (zo začiatočného stavu) vo vstupnej 
postupnosti vyskytne postupnosť 100001 (postupnosti sa môžu prekrývať, v tomto prípade 
100010001 je možné chápať ako dve postupnosti).Vlastné  riešenie overte  progr. prostriedkami 
ESPRESSO a LogiSim (príp. LOG alebo FitBoard).\\
Úlohy:
\begin{enumerate}
\item V pamäťovej časti použite minimálny počet preklápacích obvodov \textbf{JK-PO}.
\item Navrhnuté B-funkcie v tvare MDNF overte programom pre ESPRESSO. Pri návrhu 
B-funkcií klaďte dôraz na skupinovú minimalizáciu funkcií.
\item Optimálne riešenie (treba zhodnotiť, ktoré riešenie je lepšie a prečo) vytvorte obvod s 
členmi NAND (výhradne NAND, t.j. ani žiadne NOT).
\item Výslednú schému nakreslite v simulátore LogiSim (príp. LOG alebo FitBoard) 
a overte simuláciou.
\item Riešenie vyhodnoťte (zhodnotenie zadania, postup riešenia, vyjadrenie sa k počtu 
logických členov).
\end{enumerate}
\pagebreak

\section{Automat typu Moore}

\subsection{Prechodová tabuľka}
\begin{tabular}{c|c|c|c|l|}
\multirow{2}{*}{stav}&\multicolumn{2}{|c|}{Nový stav}&Y&\multirow{2}{1.3cm}{Čo je splnené?}\\ \cline{2-4}
&X=0&X=1&&\\ \hline
S0&S0&S1&0&Nič \\ \hline
S1&S2&S1&0&"1" \\ \hline
S2&S3&S1&0&"10" \\ \hline
S3&S4&S1&0&"100" \\ \hline
S4&S5&S1&0&"1000" \\ \hline
S5&S0&S6&0&"10000" \\ \hline
S6&S2&S1&1&"100001" \\ \hline
\end{tabular}

\subsection{Prechodový graf}
\begin{tikzpicture}[->,>=stealth',shorten >=1pt,node distance=2.5cm,on grid,auto] 
   \node[state,initial] (S0)   {${S0}/0$}; 
   \node[state] (S1) [right=of S0] {$S1/0$}; 
   \node[state] (S2) [above right=of S1] {$S2/0$}; 
   \node[state] (S3) [right=of S2] {$S3/0$}; 
   \node[state] (S4) [below right=of S3] {$S4/0$}; 
   \node[state,accepting] (S6) [below right=of S1] {$S6/1$}; 
   \node[state](S5) [right=of S6] {$S5/0$};
    \path[->] 
	(S0) 	edge [loop above]		node {0} 	()
	 	edge				node {1}	(S1)
	(S1) 	edge [bend left=10]	node {0} 	(S2)
	 	edge [loop above]		node {1}	()
	(S2)	edge				node {0}	(S3)
		edge [bend left=10]	node {1}	(S1)
	(S3)	edge				node {0}	(S4)
		edge [bend left=10]	node {1}	(S1)
	(S4)	edge				node {0}	(S5)
		edge [bend left=10]	node {1}	(S1)
	(S5)	edge [bend left=40]	node {0}	(S0)
		edge				node {1}	(S6)
	(S6)	edge 				node {0}	(S2)
		edge				node {1}	(S1);
\end{tikzpicture}

\subsection{Kódovanie stavov}
\begin{multicols}{2}
\begin{tabular}{c c|c|c|c|c|}
\multicolumn{4}{}{}		&\multicolumn{2}{c}{\textbf{z3}}\\ \cline{5-6}
\multicolumn{3}{}{}		&\multicolumn{2}{c}{\textbf{z2}}	&\multicolumn{1}{c}{}\\ \cline{4-5} \\ \cline{3-6}
				&						&S0	&S6	&S1	&S2\\ \cline{3-6}
\multicolumn{1}{c|}{z1}	&						&S5	&X	&S3	&S4 \\ \cline{3-6}
\end{tabular}
\begin{tabular}{|c|c|} \hline
Stav&z1z2z3\\ \hline
S0&000\\ \hline
S1&011\\ \hline
S2&001\\ \hline
S3&111\\ \hline
S4&101\\ \hline
S5&100\\ \hline
S6&010\\ \hline
\end{tabular}
\end{multicols}

\subsection{Prechodová tabuľka}
\begin{tabular}{c|c|c|c|l|}
\multirow{2}{*}{stav}&\multicolumn{2}{|c|}{Nový stav}&Y&\multirow{2}{1.3cm}{Čo je splnené?}\\ \cline{2-4}
&X=0&X=1&&\\ \hline
000&000&011&0&Nič \\ \hline
011&001&011&0&"1" \\ \hline
001&111&011&0&"10" \\ \hline
111&101&011&0&"100" \\ \hline
101&100&011&0&"1000" \\ \hline
100&000&010&0&"10000" \\ \hline
010&001&011&1&"100001" \\ \hline
\end{tabular}

\subsection{Budiace obvody pre D preklápacie obvody}
\begin{longtable}{c c}
\begin{Karnaugh}{Xz1}{z2z3}
\contingut{000,001,111,001,000,XXX,100,101,011,011,011,011,010,XXX,011,011}
\end{Karnaugh}
&
\begin{Karnaugh}{Xz1}{z2z3}
\contingut{0,0,1,0,0,X,1,1,0,0,0,0,0,X,0,0}
\end{Karnaugh}
\\
\textbf{D1,D2,D3}&\textbf{D1}\\
\begin{Karnaugh}{Xz1}{z2z3}
\contingut{0,0,1,0,0,X,0,0,1,1,1,1,1,X,1,1}
\end{Karnaugh}
&
\begin{Karnaugh}{Xz1}{z2z3}
\contingut{0,1,1,1,0,X,0,1,1,1,1,1,0,X,1,1}
\end{Karnaugh}
\\
\textbf{D2}&\textbf{D3}\\
\begin{Karnaughvuit}{z1}{z2z3}
\contingut{0,1,0,0,0,X,0,0}
\implicant{1}{5}{red}
\end{Karnaughvuit}
&\\
$\textbf{Y}=\overline{z2}.z3$&
\end{longtable}

\subsection{Budiace funkcie pre JK preklápacie obvody}
\begin{table}[h]
\centering
\begin{tabular}{|c|c|c|}\cline{2-3}
\multicolumn{1}{c|}{z->Z}		&J	&K \\ \hline
0->0					&0	&X \\ \hline
0->1					&1	&X \\ \hline
1->\textbf{\underline{0}}		&X 	&\textbf{\underline{1}} \\ \hline
1->\textbf{\underline{1}}		&X 	&\textbf{\underline{0}} \\ \hline
\end{tabular}
\end{table}
\begin{longtable}{c c}
\begin{Karnaugh}{XZ1}{Z2Z3}
\contingut{0,0,1,0,X,X,X,X,0,0,0,0,X,X,X,X}
\implicant{2}{2}{green}
\end{Karnaugh}
&
\begin{Karnaugh}{XZ1}{Z2Z3}
\contingut{X,X,X,X,1,X,0,0,X,X,X,X,1,X,1,1}
\implicant{0}{9}{blue}
\implicant{12}{10}{red}
\end{Karnaugh}
\\
$\textbf{J1}=\overline{X}.\overline{Z1}.Z2.\overline{Z3}$&$\textbf{K1}=X+\overline{Z2}$\\
\begin{Karnaugh}{XZ1}{Z2Z3}
\contingut{0,0,X,X,0,X,X,X,1,1,X,X,1,X,X,X}
\implicant{12}{10}{red}
\end{Karnaugh}
&
\begin{Karnaugh}{XZ1}{Z2Z3}
\contingut{X,X,0,1,X,X,1,1,X,X,0,0,X,X,0,0}
\implicant{1}{7}{orange}
\implicant{4}{6}{lime}
\end{Karnaugh}
\\
$\textbf{J2}=X$&$\textbf{K2}=\overline{X}.Z3+\overline{X}.\overline{Z1}$\\
\begin{Karnaugh}{XZ1}{Z2Z3}
\contingut{0,X,1,X,0,X,0,X,1,X,1,X,0,X,1,X}
\implicant{2}{2}{green}
\implicant{8}{10}{brown}
\implicant{15}{10}{purple}
\end{Karnaugh}
&
\begin{Karnaugh}{XZ1}{Z2Z3}
\contingut{X,0,X,0,X,X,X,0,X,0,X,0,X,X,X,0}
\end{Karnaugh}
\\
$\textbf{J3}=\overline{X}.\overline{Z1}.Z2.\overline{Z3}+X.Z2+X.\overline{Z1}$&$\textbf{K3}=0$\\
\end{longtable}

%\subsection{ESPRESSO}
%\subsubsection{Vstup}
%\begin{multicols}{2}
%\noindent
%.i 4 \\
%.o 7 \\
%.ilb X Z1 Z2 Z3 \\
%.ob J1 K1 J2 K2 J3 K3 Y \\
%.type fr \\
%.p 10\\
%0000 0011\\
%0001 0100\\
%0010 0101\\
%0011 0110\\
%0100 0111\\
%0101 1000\\
%0110 1001\\
%0111 1010\\
%1000 1011\\
%1001 1100\\
%.e
%\end{multicols}
%\subsubsection{Výstup}
%$A = (b\&c\&!d) | (b\&d) | (a);$\\
%$B = (b\&!c\&!d) | (!b\&c\&!d) | (!b\&d);$\\
%$C = (c\&d) | (!c\&!d);$\\
%$D = (!b\&c\&!d) | (b\&c\&!d) | (!c\&!d);$

\subsection{Prepis na NAND s využitím Shefferovej operácie}
$J1 = \overline{X}.\overline{Z1}.Z2.\overline{Z3}$\\
$J1 = \overline{\overline{\overline{X}.\overline{Z1}.Z2.\overline{Z3}+\overline{X}.\overline{Z1}.Z2.\overline{Z3}}}$\\
$J1 = \overline{(\overline{\overline{X}.\overline{Z1}.Z2.\overline{Z3}}).(\overline{\overline{X}.\overline{Z1}.Z2.\overline{Z3})}}$\\
$J1 = ((X\uparrow)\uparrow(Z1\uparrow)\uparrow Z2\uparrow(Z3\uparrow)\uparrow$\\\\
$K1 = X +\overline{Z2}$\\
$K1 = \overline{\overline{X +\overline{Z2}}}$\\
$K1 = \overline{\overline{X}.Z2}$\\
$K1 = ((X\uparrow)\uparrow Z2)$\\\\
$J2 = X$\\\\
$K2 = \overline{X}.Z3+\overline{X}.\overline{Z1}$\\
$K2 = \overline{\overline{\overline{X}.Z3+\overline{X}.\overline{Z1}}}$\\
$K2 = \overline{(\overline{\overline{X}.Z3}).(\overline{\overline{X}.\overline{Z1}})}$\\
$K2 = ((X\uparrow)\uparrow Z3)\uparrow((X\uparrow)\uparrow(Z1\uparrow))$\\\\
$J3 = \overline{X}.\overline{Z1}.Z2.\overline{Z3}+X.Z2+X.\overline{Z1}$\\
$J3 = \overline{ \overline{ \overline{X}.\overline{Z1}.Z2.\overline{Z3}+X.Z2+X.\overline{Z1}}}$\\
$J3 = \overline{(\overline{\overline{X}.\overline{Z1}.Z2.\overline{Z3}}).(\overline{X.Z2}).(\overline{X.\overline{Z1}})}$\\
$J3 = ((X\uparrow)\uparrow(Z1\uparrow)\uparrow Z2\uparrow(Z3\uparrow))\uparrow(X\uparrow Z2)\uparrow(X\uparrow(Z1\uparrow))$\\\\
$K3 = 0$\\\\
Počet logických členov NAND: ??. 
Počet preklápacích obvodov JK: ??. 
Počet vstupov do logických členov: ?? z toho ?? v kombinačnej časti a ?? v pamäťovej časti.

\end{document}