\documentclass{article}
\usepackage[english]{babel}
\usepackage[utf8]{inputenc}
\usepackage[T1]{fontenc}
\usepackage{fancyhdr}%hlavicka
\usepackage{multirow,multicol}%zjednotene riadky,stlpce v tabulke
\usepackage{amsmath}%matematicke vzorce
\usepackage{longtable}%tabulka na viac stran

\usepackage{tikz}
\usetikzlibrary{matrix,calc}

%isolated term
%#1 - Optional. Space between node and grouping line. Default=0
%#2 - node
%#3 - filling color
\newcommand{\implicantsol}[3][0]{
    \draw[rounded corners=3pt, fill=#3, opacity=0.3] ($(#2.north west)+(135:#1)$) rectangle ($(#2.south east)+(-45:#1)$);
    }


%internal group
%#1 - Optional. Space between node and grouping line. Default=0
%#2 - top left node
%#3 - bottom right node
%#4 - filling color
\newcommand{\implicant}[4][0]{
    \draw[rounded corners=3pt, fill=#4, opacity=0.3] ($(#2.north west)+(135:#1)$) rectangle ($(#3.south east)+(-45:#1)$);
    }

%group lateral borders
%#1 - Optional. Space between node and grouping line. Default=0
%#2 - top left node
%#3 - bottom right node
%#4 - filling color
\newcommand{\implicantcostats}[4][0]{
    \draw[rounded corners=3pt, fill=#4, opacity=0.3] ($(rf.east |- #2.north)+(90:#1)$)-| ($(#2.east)+(0:#1)$) |- ($(rf.east |- #3.south)+(-90:#1)$);
    \draw[rounded corners=3pt, fill=#4, opacity=0.3] ($(cf.west |- #2.north)+(90:#1)$) -| ($(#3.west)+(180:#1)$) |- ($(cf.west |- #3.south)+(-90:#1)$);
}

%group top-bottom borders
%#1 - Optional. Space between node and grouping line. Default=0
%#2 - top left node
%#3 - bottom right node
%#4 - filling color
\newcommand{\implicantdaltbaix}[4][0]{
    \draw[rounded corners=3pt, fill=#4, opacity=0.3] ($(cf.south -| #2.west)+(180:#1)$) |- ($(#2.south)+(-90:#1)$) -| ($(cf.south -| #3.east)+(0:#1)$);
    \draw[rounded corners=3pt, fill=#4, opacity=0.3] ($(rf.north -| #2.west)+(180:#1)$) |- ($(#3.north)+(90:#1)$) -| ($(rf.north -| #3.east)+(0:#1)$);
}

%group corners
%#1 - Optional. Space between node and grouping line. Default=0
%#2 - filling color
\newcommand{\implicantcantons}[2][0]{
    \draw[rounded corners=3pt, opacity=.3] ($(rf.east |- 0.south)+(-90:#1)$) -| ($(0.east |- cf.south)+(0:#1)$);
    \draw[rounded corners=3pt, opacity=.3] ($(rf.east |- 8.north)+(90:#1)$) -| ($(8.east |- rf.north)+(0:#1)$);
    \draw[rounded corners=3pt, opacity=.3] ($(cf.west |- 2.south)+(-90:#1)$) -| ($(2.west |- cf.south)+(180:#1)$);
    \draw[rounded corners=3pt, opacity=.3] ($(cf.west |- 10.north)+(90:#1)$) -| ($(10.west |- rf.north)+(180:#1)$);
    \fill[rounded corners=3pt, fill=#2, opacity=.3] ($(rf.east |- 0.south)+(-90:#1)$) -|  ($(0.east |- cf.south)+(0:#1)$) [sharp corners] ($(rf.east |- 0.south)+(-90:#1)$) |-  ($(0.east |- cf.south)+(0:#1)$) ;
    \fill[rounded corners=3pt, fill=#2, opacity=.3] ($(rf.east |- 8.north)+(90:#1)$) -| ($(8.east |- rf.north)+(0:#1)$) [sharp corners] ($(rf.east |- 8.north)+(90:#1)$) |- ($(8.east |- rf.north)+(0:#1)$) ;
    \fill[rounded corners=3pt, fill=#2, opacity=.3] ($(cf.west |- 2.south)+(-90:#1)$) -| ($(2.west |- cf.south)+(180:#1)$) [sharp corners]($(cf.west |- 2.south)+(-90:#1)$) |- ($(2.west |- cf.south)+(180:#1)$) ;
    \fill[rounded corners=3pt, fill=#2, opacity=.3] ($(cf.west |- 10.north)+(90:#1)$) -| ($(10.west |- rf.north)+(180:#1)$) [sharp corners] ($(cf.west |- 10.north)+(90:#1)$) |- ($(10.west |- rf.north)+(180:#1)$) ;
}

%Empty Karnaugh map 4x4
\newenvironment{Karnaugh}[2]%
{
\begin{tikzpicture}[baseline=(current bounding box.north),scale=0.8]
\draw (0,0) grid (4,4);
\draw (0,4) -- node [pos=0.9,above right,anchor=south west] {#2} node [pos=0.7,below left,anchor=north east] {#1} ++(135:1);
%
\matrix (mapa) [matrix of nodes,
        column sep={0.8cm,between origins},
        row sep={0.8cm,between origins},
        every node/.style={minimum size=0.3mm},
        anchor=8.center,
        ampersand replacement=\&] at (0.5,0.5)
{
                       \& |(c00)| 00         \& |(c01)| 01         \& |(c11)| 11         \& |(c10)| 10         \& |(cf)| \phantom{00} \\
|(r00)| 00             \& |(0)|  \phantom{0} \& |(1)|  \phantom{0} \& |(3)|  \phantom{0} \& |(2)|  \phantom{0} \&                     \\
|(r01)| 01             \& |(4)|  \phantom{0} \& |(5)|  \phantom{0} \& |(7)|  \phantom{0} \& |(6)|  \phantom{0} \&                     \\
|(r11)| 11             \& |(12)| \phantom{0} \& |(13)| \phantom{0} \& |(15)| \phantom{0} \& |(14)| \phantom{0} \&                     \\
|(r10)| 10             \& |(8)|  \phantom{0} \& |(9)|  \phantom{0} \& |(11)| \phantom{0} \& |(10)| \phantom{0} \&                     \\
|(rf) | \phantom{00}   \&                    \&                    \&                    \&                    \&                     \\
};
}%
{
\end{tikzpicture}
}

%Empty Karnaugh map 2x4
\newenvironment{Karnaughvuit}[2]%
{
\begin{tikzpicture}[baseline=(current bounding box.north),scale=0.8]
\draw (0,0) grid (4,2);
\draw (0,2) -- node [pos=0.9,above right,anchor=south west] {#2} node [pos=0.7,below left,anchor=north east] {#1} ++(135:1);
%
\matrix (mapa) [matrix of nodes,
        column sep={0.8cm,between origins},
        row sep={0.8cm,between origins},
        every node/.style={minimum size=0.3mm},
        anchor=4.center,
        ampersand replacement=\&] at (0.5,0.5)
{
                      \& |(c00)| 00         \& |(c01)| 01         \& |(c11)| 11         \& |(c10)| 10         \& |(cf)| \phantom{00} \\
|(r00)| 0             \& |(0)|  \phantom{0} \& |(1)|  \phantom{0} \& |(3)|  \phantom{0} \& |(2)|  \phantom{0} \&                     \\
|(r01)| 1             \& |(4)|  \phantom{0} \& |(5)|  \phantom{0} \& |(7)|  \phantom{0} \& |(6)|  \phantom{0} \&                     \\
|(rf) | \phantom{00}  \&                    \&                    \&                    \&                    \&                     \\
};
}%
{
\end{tikzpicture}
}

%Empty Karnaugh map 2x2
\newenvironment{Karnaughquatre}[2]%
{
\begin{tikzpicture}[baseline=(current bounding box.north),scale=0.8]
\draw (0,0) grid (2,2);
\draw (0,2) -- node [pos=0.7,above right,anchor=south west] {#2} node [pos=0.7,below left,anchor=north east] {#1} ++(135:1);
%
\matrix (mapa) [matrix of nodes,
        column sep={0.8cm,between origins},
        row sep={0.8cm,between origins},
        every node/.style={minimum size=0.3mm},
        anchor=2.center,
        ampersand replacement=\&] at (0.5,0.5)
{
          \& |(c00)| 0          \& |(c01)| 1  \\
|(r00)| 0 \& |(0)|  \phantom{0} \& |(1)|  \phantom{0} \\
|(r01)| 1 \& |(2)|  \phantom{0} \& |(3)|  \phantom{0} \\
};
}%
{
\end{tikzpicture}
}

%Defines 8 or 16 values (0,1,X)
\newcommand{\contingut}[1]{%
\foreach \x [count=\xi from 0]  in {#1}
     \path (\xi) node {\x};
}

%Places 1 in listed positions
\newcommand{\minterms}[1]{%
    \foreach \x in {#1}
        \path (\x) node {1};
}

%Places 0 in listed positions
\newcommand{\maxterms}[1]{%
    \foreach \x in {#1}
        \path (\x) node {0};
}

%Places X in listed positions
\newcommand{\indeterminats}[1]{%
    \foreach \x in {#1}
        \path (\x) node {X};
}%karnaughova mapa, obsahuje tikz library
\usetikzlibrary{automata,positioning,arrows}%prechodovy graf

%custom title
\usepackage{titling}
    \pretitle{\begin{center}\bfseries\fontsize{18bp}{18bp}\selectfont}
    \posttitle{\par\end{center}}
    \preauthor{}%\begin{center}\fontsize{14bp}{14bp}\selectfont}
    \author{}
    \postauthor{}%\par\end{center}\vspace{24bp}}
    \predate{}
    \date{}
    \postdate{}
    \setlength{\droptitle}{-50pt}

\title{Riešenie 3. zadania\\Syntéza sekvenčných logických obvodov}

%header definition
\pagestyle{fancy}
\fancyhf{}
\lhead{Martin Bodický}
\rhead{\today}
\cfoot{\thepage}
\renewcommand{\headrulewidth}{0.4pt}
\renewcommand{\footrulewidth}{0.4pt}

\begin{document}

%title
\maketitle
\thispagestyle{fancy}
%obsah
\section{Zadanie}
Navrhnite synchrónny sekvenčný obvod so vstupom x a výstupom y s nasledujúcim 
správaním: na výstupe y bude 1 vždy vtedy, ak sa (zo začiatočného stavu) vo vstupnej 
postupnosti vyskytne postupnosť 100001 (postupnosti sa môžu prekrývať, v tomto prípade 
100010001 je možné chápať ako dve postupnosti).Vlastné  riešenie overte  progr. prostriedkami 
ESPRESSO a LogiSim (príp. LOG alebo FitBoard).\\
Úlohy:
\begin{enumerate}
\item V pamäťovej časti použite minimálny počet preklápacích obvodov \textbf{JK-PO}.
\item Navrhnuté B-funkcie v tvare MDNF overte programom pre ESPRESSO. Pri návrhu 
B-funkcií klaďte dôraz na skupinovú minimalizáciu funkcií.
\item Optimálne riešenie (treba zhodnotiť, ktoré riešenie je lepšie a prečo) vytvorte obvod s 
členmi NAND (výhradne NAND, t.j. ani žiadne NOT).
\item Výslednú schému nakreslite v simulátore LogiSim (príp. LOG alebo FitBoard) 
a overte simuláciou.
\item Riešenie vyhodnoťte (zhodnotenie zadania, postup riešenia, vyjadrenie sa k počtu 
logických členov).
\end{enumerate}
\pagebreak

\section{Automat typu Moore}

\subsection{Prechodová tabuľka}
\begin{tabular}{c|c|c|c|l|}
\multirow{2}{*}{stav}&\multicolumn{2}{|c|}{Nový stav}&Y&\multirow{2}{1.3cm}{Čo je splnené?}\\ \cline{2-4}
&X=0&X=1&&\\ \hline
S0&S0&S1&0&Nič \\ \hline
S1&S2&S1&0&"1" \\ \hline
S2&S3&S1&0&"10" \\ \hline
S3&S4&S1&0&"100" \\ \hline
S4&S5&S1&0&"1000" \\ \hline
S5&S0&S6&0&"10000" \\ \hline
S6&S2&S1&1&"100001" \\ \hline
\end{tabular}

\subsection{Prechodový graf}
\begin{tikzpicture}[->,>=stealth',shorten >=1pt,node distance=2.5cm,on grid,auto] 
   \node[state,initial] (S0)   {$S0/0$}; 
   \node[state] (S1) [right=of S0] {$S1/0$}; 
   \node[state] (S2) [above right=of S1] {$S2/0$}; 
   \node[state] (S3) [right=of S2] {$S3/0$}; 
   \node[state] (S4) [below right=of S3] {$S4/0$}; 
   \node[state,accepting] (S6) [below right=of S1] {$S6/1$}; 
   \node[state](S5) [right=of S6] {$S5/0$};
    \path[->] 
	(S0) 	edge [loop above]		node {0} 	()
	 	edge				node {1}	(S1)
	(S1) 	edge [bend left=10]	node {0} 	(S2)
	 	edge [loop above]		node {1}	()
	(S2)	edge				node {0}	(S3)
		edge [bend left=10]	node {1}	(S1)
	(S3)	edge				node {0}	(S4)
		edge [bend left=10]	node {1}	(S1)
	(S4)	edge				node {0}	(S5)
		edge [bend left=10]	node {1}	(S1)
	(S5)	edge [bend left=40]	node {0}	(S0)
		edge				node {1}	(S6)
	(S6)	edge 				node {0}	(S2)
		edge				node {1}	(S1);
\end{tikzpicture}

\subsection{Kódovanie stavov}
\begin{multicols}{2}
\begin{tabular}{c c|c|c|c|c|}
\multicolumn{4}{}{}		&\multicolumn{2}{c}{\textbf{z3}}\\ \cline{5-6}
\multicolumn{3}{}{}		&\multicolumn{2}{c}{\textbf{z2}}	&\multicolumn{1}{c}{}\\ \cline{4-5} \\ \cline{3-6}
				&						&S0	&S6	&S1	&S2\\ \cline{3-6}
\multicolumn{1}{c|}{z1}	&						&S5	&X	&S3	&S4 \\ \cline{3-6}
\end{tabular}

\begin{tabular}{|c|c|} \hline
Stav&z1z2z3\\ \hline
S0&000\\ \hline
S1&011\\ \hline
S2&001\\ \hline
S3&111\\ \hline
S4&101\\ \hline
S5&100\\ \hline
S6&010\\ \hline
\end{tabular}
\end{multicols}

\subsection{Prechodová tabuľka}
\begin{tabular}{c|c|c|c|l|}
\multirow{2}{*}{stav}&\multicolumn{2}{|c|}{Nový stav}&Y&\multirow{2}{1.3cm}{Čo je splnené?}\\ \cline{2-4}
&X=0&X=1&&\\ \hline
000&000&011&0&Nič \\ \hline
011&001&011&0&"1" \\ \hline
001&111&011&0&"10" \\ \hline
111&101&011&0&"100" \\ \hline
101&100&011&0&"1000" \\ \hline
100&000&010&0&"10000" \\ \hline
010&001&011&1&"100001" \\ \hline
\end{tabular}

\subsection{Budiace obvody pre D preklápacie obvody}
\begin{longtable}{c c}
\begin{Karnaugh}{Xz1}{z2z3}
\contingut{000,111,001,001,000,100,XXX,101,011,011,011,011,010,011,XXX,011}
\end{Karnaugh}
&
\begin{Karnaugh}{Xz1}{z2z3}
\contingut{0,1,0,0,0,1,X,1,0,0,0,0,0,0,X,0}
\end{Karnaugh}
\\
\textbf{D1,D2,D3}&\textbf{D1}\\
\begin{Karnaugh}{Xz1}{z2z3}
\contingut{0,1,0,0,0,0,X,0,1,1,1,1,1,1,X,1}
\end{Karnaugh}
&
\begin{Karnaugh}{Xz1}{z2z3}
\contingut{0,1,1,1,0,0,X,1,1,1,1,1,0,1,X,1}
\end{Karnaugh}
\\
\textbf{D2}&\textbf{D3}\\
\begin{Karnaughvuit}{z1}{z2z3}
\contingut{0,0,1,0,0,0,X,0}
\implicant{2}{6}{red}
\end{Karnaughvuit}
&\\
$\textbf{Y}=z2.\overline{z3}$&
\end{longtable}

\subsection{Budiace funkcie pre JK preklápacie obvody}
\begin{table}[h]
\centering
\begin{tabular}{|c|c|c|}\cline{2-3}
\multicolumn{1}{c|}{z->Z}		&J	&K \\ \hline
0->0					&0	&X \\ \hline
0->1					&1	&X \\ \hline
1->\textbf{\underline{0}}		&X 	&\textbf{\underline{1}} \\ \hline
1->\textbf{\underline{1}}		&X 	&\textbf{\underline{0}} \\ \hline
\end{tabular}
\end{table}
\begin{longtable}{c c}
\begin{Karnaugh}{XZ1}{Z2Z3}
\contingut{0,1,0,0,X,X,X,X,0,0,0,0,X,X,X,X}
\implicantsol{1}{green}
\end{Karnaugh}
&
\begin{Karnaugh}{XZ1}{Z2Z3}
\contingut{X,X,X,X,1,0,X,0,X,X,X,X,1,1,X,1}
\implicant{12}{10}{blue}
\implicantcostats{0}{10}{red}
\end{Karnaugh}
\\
$\textbf{J1}=\overline{X}.\overline{Z1}.\overline{Z2}.Z3$&$\textbf{K1}=X+\overline{Z3}$\\
\begin{Karnaugh}{XZ1}{Z2Z3}
\contingut{0,1,X,X,0,0,X,X,1,1,X,X,1,1,X,X}
\implicantsol{1}{green}
\implicant{12}{10}{blue}
\end{Karnaugh}
&
\begin{Karnaugh}{XZ1}{Z2Z3}
\contingut{X,X,1,1,X,X,X,1,X,X,0,0,X,X,X,0}
\implicant{0}{6}{orange}
\end{Karnaugh}
\\
$\textbf{J2}=X+\overline{X}.\overline{Z1}.\overline{Z2}.Z3$&$\textbf{K2}=\overline{X}$\\
\begin{Karnaugh}{XZ1}{Z2Z3}
\contingut{0,X,1,X,0,X,X,X,1,X,1,X,0,X,X,X}
\implicant{3}{10}{yellow}
\implicant{8}{10}{violet}
\end{Karnaugh}
&
\begin{Karnaugh}{XZ1}{Z2Z3}
\contingut{X,0,X,0,X,1,X,0,X,0,X,0,X,0,X,0}
\implicant{4}{5}{pink}
\end{Karnaugh}
\\
$\textbf{J3}=Z2+X.\overline{Z1}$&$\textbf{K3}=\overline{X}.Z1.\overline{Z2}$\\
\end{longtable}

\subsection{ESPRESSO}
\subsubsection{Vstup}
\begin{multicols}{2}
\noindent
.i 4\\
.o 7\\
.ilb X Z1 Z2 Z3\\
.ob J1 K1 J2 K2 J3 K3 Y\\
.type fr\\
.p 16\\
0000 0-0-0-0\\
0001 1-1--00\\
0010 0--11-1\\
0011 0--1-00\\
0100 -10-0-0\\
0101 -00--10\\
0110 ------\\
0111 -0-1-00\\
1000 0-1-1-0\\
1001 0-1--00\\
1010 0--01-1\\
1011 0--0-00\\
1100 -11-0-0\\
1101 -11--00\\
1110 -------\\
1111 -1-0-00\\
.e\\

\end{multicols}
\subsubsection{Výstup}
$J1 = (!X\&!Z1\&!Z2\&Z3);$\\
$K1 = (!Z3) | (X);$\\
$J2 = (!X\&!Z1\&!Z2\&Z3) | (X);$\\
$K2 = (!X);$\\
$J3 = (X\&!Z1) | (Z2\&!Z3);$\\
$K3 = (!X\&Z1\&!Z2);$\\
$Y = (Z2\&!Z3);$\\

\subsection{Prepis na NAND s využitím Shefferovej operácie}
$J1 = \overline{X}.\overline{Z1}.\overline{Z2}.Z3$\\
$J1 = \overline{\overline{\overline{X}.\overline{Z1}.\overline{Z2}.Z3+\overline{X}.\overline{Z1}.\overline{Z2}.Z3}}$\\
$J1 = \overline{(\overline{\overline{X}.\overline{Z1}.\overline{Z2}.Z3}).(\overline{\overline{X}.\overline{Z1}.\overline{Z2}.Z3})}$\\
$J1 = ((X\uparrow)\uparrow(Z1\uparrow)\uparrow(Z2\uparrow)\uparrow Z3)\uparrow$\\\\
$K1 = X+\overline{Z3}$\\
$K1 = \overline{\overline{X+\overline{Z3}}}$\\
$K1 = \overline{\overline{X}.Z3}$\\
$K1 = (X\uparrow)\uparrow Z3$\\\\
$J2 = X+\overline{X}.\overline{Z1}.\overline{Z2}.Z3$\\
$J2 = \overline{\overline{X+\overline{X}.\overline{Z1}.\overline{Z2}.Z3}}$\\
$J2 = \overline{\overline{X}.(\overline{\overline{X}.\overline{Z1}.\overline{Z2}.Z3})}$\\
$J2 = (X\uparrow)\uparrow((X\uparrow)\uparrow(Z1\uparrow)\uparrow(Z2\uparrow)\uparrow Z3)$\\\\
$K2 = \overline{X}$\\
$K2 = X\uparrow$\\\\
$J3 = Z2+X.\overline{Z1}$\\
$J3 = \overline{\overline{Z2+X.\overline{Z1}}}$\\
$J3 = \overline{\overline{Z2}.(\overline{X.\overline{Z1}})}$\\
$J3 = (Z2\uparrow)\uparrow(X\uparrow(Z1\uparrow))$\\\\
$K3 = \overline{X}.Z1.\overline{Z2}$\\
$K3 = \overline{\overline{\overline{X}.Z1.\overline{Z2}+\overline{X}.Z1.\overline{Z2}}}$\\
$K3 = \overline{(\overline{\overline{X}.Z1.\overline{Z2}}).(\overline{\overline{X}.Z1.\overline{Z2}})}$\\
$K3 = ((X\uparrow)\uparrow Z1\uparrow(Z2\uparrow))\uparrow$\\\\
$Y = Z2.\overline{Z3}$\\
$Y = \overline{\overline{Z2.\overline{Z3}+Z2.\overline{Z3}}}$\\
$Y = \overline{(\overline{Z2.\overline{Z3}}).(\overline{Z2.\overline{Z3}})}$\\
$Y = (Z2\uparrow(Z3\uparrow))\uparrow$\\\\
Počet logických členov NAND: 11. 
Počet preklápacích obvodov JK: 3. 
Počet vstupov do logických členov: 37 z toho 25 v kombinačnej časti a 12 v pamäťovej časti.\pagebreak


\section{Automat typu Mealy}

\subsection{Prechodová tabuľka}
\begin{tabular}{c|c|c|c|c|l|}
\multirow{2}{*}{stav}&\multicolumn{2}{|c|}{Nový stav}&\multicolumn{2}{|c|}{Y}&\multirow{2}{1.3cm}{Čo je splnené?}\\ \cline{2-5}
&X=0&X=1&X=0&X=1&\\ \hline
S0&S0&S1&0&0&Nič \\ \hline
S1&S2&S1&0&0&"1" \\ \hline
S2&S3&S1&0&0&"10" \\ \hline
S3&S4&S1&0&0&"100" \\ \hline
S4&S5&S1&0&0&"1000" \\ \hline
S5&S0&S1&0&1&"10000" \\ \hline
\end{tabular}

\subsection{Prechodový graf}
\begin{tikzpicture}[->,>=stealth',shorten >=1pt,node distance=2.5cm,on grid,auto] 
   \node[state,initial] (S0)   {$S0$}; 
   \node[state] (S1) [right=of S0] {$S1$}; 
   \node[state] (S2) [right=of S1] {$S2$}; 
   \node[state] (S3) [below=of S2] {$S3$}; 
   \node[state] (S4) [left=of S3] {$S4$}; 
   \node[state](S5) [left=of S4] {$S5$};
    \path[->] 
	(S0) 	edge [loop above]		node {0/0} 	()
	 	edge				node {1/0}	(S1)
	(S1) 	edge [bend left=10]	node {0/0} 	(S2)
	 	edge [loop above]		node {1/0}	()
	(S2)	edge				node {0/0}	(S3)
		edge [bend left=10]	node {1/0}	(S1)
	(S3)	edge				node {0/0}	(S4)
		edge 				node {1/0}	(S1)
	(S4)	edge				node {0/0}	(S5)
		edge 				node {1/0}	(S1)
	(S5)	edge 				node {0/0}	(S0)
		edge 				node {1/1}	(S1);
\end{tikzpicture}

\subsection{Kódovanie stavov}
\begin{multicols}{2}
\begin{tabular}{c c|c|c|c|c|}
\multicolumn{4}{}{}		&\multicolumn{2}{c}{\textbf{z3}}\\ \cline{5-6}
\multicolumn{3}{}{}		&\multicolumn{2}{c}{\textbf{z2}}	&\multicolumn{1}{c}{}\\ \cline{4-5} \\ \cline{3-6}
				&						&S0	&S1	&S2	&X\\ \cline{3-6}
\multicolumn{1}{c|}{z1}	&						&S5	&S4	&S3	&X \\ \cline{3-6}
\end{tabular}

\begin{tabular}{|c|c|} \hline
Stav&z1z2z3\\ \hline
S0&000\\ \hline
S1&010\\ \hline
S2&011\\ \hline
S3&111\\ \hline
S4&110\\ \hline
S5&100\\ \hline
\end{tabular}
\end{multicols}

\subsection{Prechodová tabuľka}
\begin{tabular}{c|c|c|c|c|l|}
\multirow{2}{*}{stav}&\multicolumn{2}{|c|}{Nový stav}&\multicolumn{2}{|c|}{Y}&\multirow{2}{1.3cm}{Čo je splnené?}\\ \cline{2-5}
&X=0&X=1&X=0&X=1&\\ \hline
000&000&010&0&0&Nič \\ \hline
010&011&010&0&0&"1" \\ \hline
011&111&010&0&0&"10" \\ \hline
111&110&010&0&0&"100" \\ \hline
110&100&010&0&0&"1000" \\ \hline
100&000&010&0&1&"10000" \\ \hline
\end{tabular}

\subsection{Budiace obvody pre D preklápacie obvody}
\begin{longtable}{c c}
\begin{Karnaugh}{Xz1}{z2z3}
\contingut{000,XXX,011,111,000,XXX,100,110,010,XXX,010,010,010,XXX,010,010}
\end{Karnaugh}
&
\begin{Karnaugh}{Xz1}{z2z3}
\contingut{0,X,0,1,0,X,1,1,0,X,0,0,0,X,0,0}
\end{Karnaugh}
\\
\textbf{D1,D2,D3}&\textbf{D1}\\
\begin{Karnaugh}{Xz1}{z2z3}
\contingut{0,X,1,1,0,X,0,1,1,X,1,1,1,X,1,1}
\end{Karnaugh}
&
\begin{Karnaugh}{Xz1}{z2z3}
\contingut{0,X,1,1,0,X,0,0,0,X,0,0,0,X,0,0}
\end{Karnaugh}
\\
\textbf{D2}&\textbf{D3}\\
\begin{Karnaugh}{Xz1}{z2z3}
\contingut{0,X,0,0,0,X,0,0,0,X,0,0,1,X,0,0}
\implicant{12}{13}{red}
\end{Karnaugh}
&\\
$\textbf{Y}=X.z1.\overline{z2}$&
\end{longtable}

\subsection{Budiace funkcie pre JK preklápacie obvody}
\begin{table}[h]
\centering
\begin{tabular}{|c|c|c|}\cline{2-3}
\multicolumn{1}{c|}{z->Z}		&J	&K \\ \hline
0->0					&0	&X \\ \hline
0->1					&1	&X \\ \hline
1->\textbf{\underline{0}}		&X 	&\textbf{\underline{1}} \\ \hline
1->\textbf{\underline{1}}		&X 	&\textbf{\underline{0}} \\ \hline
\end{tabular}
\end{table}
\begin{longtable}{c c}
\begin{Karnaugh}{XZ1}{Z2Z3}
\contingut{0,X,0,1,X,X,X,X,0,X,0,0,X,X,X,X}
\implicant{1}{7}{green}
\end{Karnaugh}
&
\begin{Karnaugh}{XZ1}{Z2Z3}
\contingut{X,X,X,X,1,X,0,0,X,X,X,X,1,X,1,1}
\implicant{12}{10}{red}
\implicant{0}{9}{blue}
\end{Karnaugh}
\\
$\textbf{J1}=\overline{X}.Z3$&$\textbf{K1}=X+\overline{Z2}$\\
\begin{Karnaugh}{XZ1}{Z2Z3}
\contingut{0,X,X,X,0,X,X,X,1,X,X,X,1,X,X,X}
\implicant{12}{10}{red}
\end{Karnaugh}
&
\begin{Karnaugh}{XZ1}{Z2Z3}
\contingut{X,X,0,0,X,X,1,0,X,X,0,0,X,X,0,0}
\implicantcostats{4}{6}{yellow}
\end{Karnaugh}
\\
$\textbf{J2}=X$&$\textbf{K2}=\overline{X}.Z1.\overline{Z3}$\\
\begin{Karnaugh}{XZ1}{Z2Z3}
\contingut{0,X,1,X,0,X,0,X,0,X,0,X,0,X,0,X}
\implicant{3}{2}{orange}
\end{Karnaugh}
&
\begin{Karnaugh}{XZ1}{Z2Z3}
\contingut{X,X,X,0,X,X,X,1,X,X,X,1,X,X,X,1}
\implicant{12}{10}{red}
\implicant{4}{14}{lime}
\end{Karnaugh}
\\
$\textbf{J3}=\overline{X}.\overline{Z1}.Z2$&$\textbf{K3}=X+Z1$\\
\end{longtable}

\subsection{ESPRESSO}
\subsubsection{Vstup}
\begin{multicols}{2}
\noindent
.i 4\\
.o 7\\
.ilb X Z1 Z2 Z3\\
.ob J1 K1 J2 K2 J3 K3 Y\\
.type fr\\
.p 16\\
0000 0-0-0-0\\
0001 -------\\
0010 0--01-0\\
0011 1--0-00\\
0100 -10-0-0\\
0101 -------\\
0110 -0-10-0\\
0111 -0-0-10\\
1000 0-1-0-0\\
1001 -------\\
1010 0--00-0\\
1011 0--0-10\\
1100 -11-0-1\\
1101 -------\\
1110 -1-00-0\\
1111 -1-0-10\\
.e\\

\end{multicols}
\subsubsection{Výstup}
$J1 = (!X\&Z3);$\\
$K1 = (!Z2) | (X);$\\
$J2 = (X);$\\
$K2 = (!X\&Z1\&!Z3);$\\
$J3 = (!X\&!Z1\&Z2);$\\
$K3 = (Z1) | (X);$\\
$Y = (X\&Z1\&!Z2);$\\

\subsection{Prepis na NAND s využitím Shefferovej operácie}
$J1 = \overline{X}.Z3$\\
$J1 = \overline{\overline{\overline{X}.Z3+\overline{X}.Z3}}$\\
$J1 = \overline{(\overline{\overline{X}.Z3}).(\overline{\overline{X}.Z3})}$\\
$J1 = ((X\uparrow)\uparrow Z3)\uparrow$\\\\
$K1 = X+\overline{Z2}$\\
$K1 = \overline{\overline{X+\overline{Z2}}}$\\
$K1 = \overline{\overline{X}.Z2}$\\
$K1 = (X\uparrow)\uparrow Z2$\\\\
$J2 = X$\\\\
$K2 = \overline{X}.Z1.\overline{Z3}$\\
$K2 = \overline{\overline{\overline{X}.Z1.\overline{Z3}+\overline{X}.Z1.\overline{Z3}}}$\\
$K2 = \overline{(\overline{\overline{X}.Z1.\overline{Z3}}).(\overline{\overline{X}.Z1.\overline{Z3}})}$\\
$K2 = ((X\uparrow)\uparrow Z1\uparrow(Z3\uparrow))\uparrow$\\\\
$J3 = \overline{X}.\overline{Z1}.Z2$\\
$J3 = \overline{\overline{\overline{X}.\overline{Z1}.Z2+\overline{X}.\overline{Z1}.Z2}}$\\
$J3 = \overline{(\overline{X}.\overline{Z1}.Z2).(\overline{X}.\overline{Z1}.Z2)}$\\
$J3 = ((X\uparrow)\uparrow(Z1\uparrow)\uparrow Z2)\uparrow$\\\\
$K3 = X+Z1$\\
$K3 = \overline{\overline{X+Z1}}$\\
$K3 = \overline{\overline{X}.\overline{Z1}}$\\
$K3 = (X\uparrow)\uparrow(Z1\uparrow)$\\\\
$Y = X.Z1.\overline{Z2}$\\
$Y = \overline{\overline{X.Z1.\overline{Z2}+X.Z1.\overline{Z2}}}$\\
$Y = \overline{(\overline{X.Z1.\overline{Z2}}).(\overline{X.Z1.\overline{Z2}})}$\\
$Y = (X\uparrow Z1\uparrow(Z2\uparrow))\uparrow$\\\\
Počet logických členov NAND: 11. 
Počet preklápacích obvodov JK: 3. 
Počet vstupov do logických členov: 37 z toho 25 v kombinačnej časti a 12 v pamäťovej časti.

\section{Zhodnotenie}
Mojou úlohou bolo vytvoriť automat typu Moore alebo Mealy na sledovanie postupnosti bitov 100001. Pri riešení som počítal s prekrívaním prvého a posledného bitu. Riešenie som spravil pre oba automaty. 

Pri ich tvorbe som si zakreslil graf s prechodmi medzi stavmi a následne prechodovú tabuľku. Pri kódovaní stavov som sa ich snažil zakódovať čo najoptimálnejšie a to tak aby sa pri prechode medzi dvomi stavmi menilo čo najmenej bitov uložených v JK-PO. Následne som prechodovú tabuľku prepísal do Karnaughovej mapy ako budiacie funkcie pre D-PO, ktoré som previedol na JK-PO.

Moje riešenie MDNF pre automat Mealy je zhodné s výstupom z ESPRESSO a pre automat Moore je tam len jedna nepodstatná odlišnosť. Ja som navrhol $\textbf{J3}=Z2+X.\overline{Z1}$, a ESPRESSO použilo skupinovú minimalizáciu s $\textbf{Y}=Z2.\overline{Z3}$. Kde $Z2$ z $\textbf{J3}$ nahradilo $\textbf{Y}$ takto $\textbf{J3}=Z2.\overline{Z3}+X.\overline{Z1}$.

Pri prepise MDNF na Shefferovu operáciu som použil dvojitú negáciu a De Morganove pravidlo. Oba obvody som zakreslil pomocou logických členov NAND v programe LOGISIM. Pri oboch automatoch som potreboval rovnaký počet logických členov ako aj JK-PO a aj identický počet vstupov do logických členov. Na simuláciu obvodu som použil v LOGISIMe aj komponenty ROM v ktorom je uložená sekvencia 0 a 1 a Counter ktorý postupne posiela jednotlivé bity z ROM do automatov Moore a Mealy. Podrobnejšie informácie sú v LOGISIMe.

\end{document}