\documentclass{article}
\usepackage[english]{babel}
\usepackage[utf8]{inputenc}
\usepackage[T1]{fontenc}
\usepackage{fancyhdr}
\usepackage{multirow,multicol}
\usepackage{amsmath}
\usepackage{longtable}
\usepackage{multicol}

\input{karnaugh}

%custom title
\usepackage{titling}
    \pretitle{\begin{center}\bfseries\fontsize{18bp}{18bp}\selectfont}
    \posttitle{\par\end{center}}
    \preauthor{}%\begin{center}\fontsize{14bp}{14bp}\selectfont}
    \author{}
    \postauthor{}%\par\end{center}\vspace{24bp}}
    \predate{}
    \date{}
    \postdate{}
    \setlength{\droptitle}{-50pt}

\title{Riešenie 2. zadania\\Syntéza kombinačných logických obvodov}

%header definition
\pagestyle{fancy}
\fancyhf{}
\lhead{Martin Bodický}
\rhead{\today}
\cfoot{\thepage}
\renewcommand{\headrulewidth}{0.4pt}
\renewcommand{\footrulewidth}{0.4pt}

\begin{document}

%title
\maketitle
\thispagestyle{fancy}
%obsah
\section{Zadanie}
Navrhnite prevodník číslic  0-9 v kóde BCD8421 do kódu BCD8421+3. Prevodník realizujte s 
minimálnym počtom členov NAND a NOR. Vlastné  riešenie overte  progr. prostriedkami ESPRESSO a 
LogiSim (príp. LOG alebo FitBoard). 
Úlohy:
\begin{enumerate}
\item Navrhnite vlastné riešenie skupinovej minimalizácie
 a odvoďte B-funkcie v tvare MDNF.
\item Vytvorte vstupný textový súbor s opisom vstupu pre ESPRESSO.
\item Navrhnuté B-funkcie v tvare MDNF overte programom pre ESPRESSO. Pri návrhu 
B-funkcií klaďte dôraz na skupinovú minimalizáciu funkcií.
\item Optimálne riešenie (treba zhodnotiť, ktoré riešenie
je lepšie a prečo) vytvorte obvod s členmi NAND (výhradne NAND, t.j. ani žiadne NOT).
\item Z Karnaughovej mapy odvoďte B-funkcie v tvare MKNF 
a vytvorte obvod s členmi NOR (výhradne NOR, t.j. ani žiadne NOT). 
\item Výslednú schému nakreslite v simulátore LogiSim (príp. LOG alebo FitBoard) 
a overte simuláciou. 
\item Riešenie vyhodnoťte (zhodnotenie zadania, postup riešenia, vyjadrenie sa k počtu 
logických členov, vstupov obvodu, vhodnosti použitie NAND alebo NOR realizácie). 
\end{enumerate}
\pagebreak

\section{Tabuľka funkčných hodnôt a Karnaughova mapa}
\begin{tabular}{c|c c c c|c c c c}
\#&\multicolumn{4}{c|}{\textbf{BCD8421}}&\multicolumn{4}{c}{\textbf{BCD8421+3}}\\ \hline
0&0&0&0&0&0&0&1&1\\ \hline
1&0&0&0&1&0&1&0&0\\ \hline
2&0&0&1&0&0&1&0&1\\ \hline
3&0&0&1&1&0&1&1&0\\ \hline
4&0&1&0&0&0&1&1&1\\ \hline
5&0&1&0&1&1&0&0&0\\ \hline
6&0&1&1&0&1&0&0&1\\ \hline
7&0&1&1&1&1&0&1&0\\ \hline
8&1&0&0&0&1&0&1&1\\ \hline
9&1&0&0&1&1&1&0&0
\end{tabular}
\begin{tabular}{c c c|c|c|c}
\multicolumn{4}{}{}&\multicolumn{2}{c}{\textbf{c}}\\ \cline{5-6}
\multicolumn{3}{}{}&\multicolumn{2}{c}{\textbf{d}}&\\ \cline{4-5} \\
&&0011&0100&0110&0101\\
&\multicolumn{1}{c|}{\multirow{2}{*}{b}}&0111&1000&1010&1001\\
\multicolumn{1}{c|}{\multirow{2}{*}{a}}&\multicolumn{1}{c|}{}&XXXX&XXXX&XXXX&XXXX\\
\multicolumn{1}{c|}{}&&1011&1100&XXXX&XXXX
\end{tabular}

\section{Karnaughove mapy a DNF}
\subsection{Mapy}
\begin{longtable}{c c}
\begin{Karnaugh}
\contingut{0,0,0,0,0,1,1,1,1,1,X,X,X,X,X,X}
\implicant{12}{10}{red}
\implicant{5}{15}{green}
\implicant{7}{14}{blue}
\end{Karnaugh}
&
\begin{Karnaugh}
\contingut{0,1,1,1,1,0,0,0,0,1,X,X,X,X,X,X}
\implicantdaltbaix{1}{11}{cyan}
\implicantdaltbaix{3}{10}{yellow}
\implicant{4}{12}{brown}
\end{Karnaugh}
\\
\textbf{A}&\textbf{B}\\
\begin{Karnaugh}
\contingut{1,0,0,1,1,0,0,1,1,0,X,X,X,X,X,X}
\implicant{0}{8}{orange}
\implicant{3}{11}{magenta}
\end{Karnaugh}
&
\begin{Karnaugh}
\contingut{1,0,1,0,1,0,1,0,1,0,X,X,X,X,X,X}
\implicantcostats{0}{10}{lime}
\end{Karnaugh}
\\
\textbf{C}&\textbf{D}
\end{longtable}

\subsection{MDNF}
$A = a + bc + bd$\\
$B = \overline{b}c + \overline{b}d + b\overline{c}\overline{d}$\\
$C = cd + \overline{c}\overline{d}$\\
$D = \overline{d}$

\subsection{ESPRESSO}
\subsubsection{Vstup}
\begin{multicols}{2}
\noindent
.i 4 \\
.o 4 \\
.ilb a b c d \\
.ob A B C D \\
.type fr \\
.p 10\\
0000 0011\\
0001 0100\\
0010 0101\\
0011 0110\\
0100 0111\\
0101 1000\\
0110 1001\\
0111 1010\\
1000 1011\\
1001 1100\\
.e
\end{multicols}
\subsubsection{Výstup}
$A = (b\&c\&!d) | (b\&d) | (a);$\\
$B = (b\&!c\&!d) | (!b\&c\&!d) | (!b\&d);$\\
$C = (c\&d) | (!c\&!d);$\\
$D = (!b\&c\&!d) | (b\&c\&!d) | (!c\&!d);$

\subsection{Prepis na NAND}
$A = a + bc + bd$\\
$A = \overline{\overline{a + bc + bd}}$\\
$A = \overline{a}(\overline{bc})(\overline{bd})$\\
$A = (a\uparrow)\uparrow(b\uparrow c)\uparrow(b\uparrow d)$\\\\
$B = \overline{b}c + \overline{b}d + b\overline{c}\overline{d}$\\
$B = \overline{\overline{\overline{b}c + \overline{b}d + b\overline{c}\overline{d}}}$\\
$B = \overline{\overline{(\overline{b}c})(\overline{\overline{b}d})(\overline{b\overline{c}\overline{d}})}$\\
$B = ((b\uparrow)\uparrow c)\uparrow((b\uparrow)\uparrow d)\uparrow(b\uparrow(c\uparrow)\uparrow(d\uparrow))$\\\\
$C = cd + \overline{c}\overline{d}$\\
$C = \overline{\overline{cd + \overline{c}\overline{d}}}$\\
$C = \overline{(\overline{cd})(\overline{\overline{c}\overline{d}})}$\\
$C = (c \uparrow d)\uparrow((c\uparrow)\uparrow(d\uparrow))$\\\\
$D = \overline{d}$\\
$D = \overline{\overline{\overline{d}}}$\\
$D = \overline{d}$\\
$D = d\uparrow$\\\\
Počet logických členov: 14\\
Počet vstupov do logických členov: 31


\section{Karnaughove mapy a KNF}
\subsection{Mapy}
\begin{longtable}{c c}
\begin{Karnaugh}
\contingut{0,0,0,0,0,1,1,1,1,1,X,X,X,X,X,X}
\implicant{0}{2}{red}
\implicant{0}{4}{green}
\end{Karnaugh}
&
\begin{Karnaugh}
\contingut{0,1,1,1,1,0,0,0,0,1,X,X,X,X,X,X}
\implicant{7}{14}{blue}
\implicant{5}{15}{yellow}
\implicantdaltbaix{0}{8}{brown}
\end{Karnaugh}
\\
\textbf{A}&\textbf{B}\\
\begin{Karnaugh}
\contingut{1,0,0,1,1,0,0,1,1,0,X,X,X,X,X,X}
\implicant{1}{9}{cyan}
\implicant{2}{10}{magenta}
\end{Karnaugh}
&
\begin{Karnaugh}
\contingut{1,0,1,0,1,0,1,0,1,0,X,X,X,X,X,X}
\implicant{1}{11}{orange}
\end{Karnaugh}
\\
\textbf{C}&\textbf{D}
\end{longtable}

\subsection{MKNF}
$A = (a+b)(b+c+d)$\\
$B = (\overline{b}+\overline{c})(\overline{b}+\overline{d})(b+c+d)$\\
$C = (c+\overline{d})(\overline{c}+d)$\\
$D = \overline{d}$

\subsection{Prepis na NOR}
$A = (a+b)(b+c+d)$\\
$A = \overline{\overline{(a+b)(b+c+d)}}$\\
$A = \overline{(\overline{a+b})+(\overline{a+c+d})}$\\
$A = (a\downarrow b)\downarrow(a\downarrow c\downarrow d)$\\\\
$B = (\overline{b}+\overline{c})(\overline{b}+\overline{d})(b+c+d)$\\
$B = \overline{\overline{(\overline{b}+\overline{c})(\overline{b}+\overline{d})(b+c+d)}}$\\
$B = \overline{(\overline{\overline{b}+\overline{c}})+(\overline{\overline{b}+\overline{d}})+(\overline{b+c+d})}$\\
$B = ((b\downarrow)\downarrow(c\downarrow))\downarrow((b\downarrow)\downarrow(d\downarrow))\downarrow(b\downarrow c\downarrow d)$\\\\
$C = (c+\overline{d})(\overline{c}+d)$\\
$C = \overline{\overline{(c+\overline{d})(\overline{c}+d)}}$\\
$C = \overline{(\overline{c+\overline{d}})+(\overline{\overline{c}+d})}$\\
$C = (c\downarrow(d\downarrow))\downarrow((c\downarrow)\downarrow d)$\\\\
$D = \overline{d}$\\
$D = \overline{\overline{\overline{d}}}$\\
$D = \overline{d}$\\
$D = d\downarrow$\\\\
Počet logických členov: 14\\
Počet vstupov do logických členov: 31

\section{Zhodnotenie}
Mojou úlohou bolo vytvoriť obvdod na prevod číslic z BCD8421 do BCD8421+3. V MDNF aj v MKNF mi vyšiel počet logických členov 14 a počet vstupov do logických členov 31. Z ESPRESSa my vyšiel výstup potrebujúci 14 logických členov a 34 vstupov do logických členov. ESPRESSO síce zjednotilo pre niektoré výstupy logické členy, ale pribudli 3 vstupy do logických členov. V tomto prípade je jedno, či by sa obvod realizoval pomocou členov NAND alebo NOR, pretože oba obvody potrebujú rovnaký počet logických členov aj rovnaký počet vstupov do logických členov.


\end{document}